\section{Modelo de Entidad Relación y Modelo Relacional}

A continuación, se mostrarán el Modelo de Entidad Relación y el Modelo Relacional correspondientes a nuestra solución del problema.

\subsection{Modelo de Entidad Relación}

\subsection{Modelo Relacional}

Restricciones generales:

\begin{itemize}

	\item la graduació en todos los casos es cinturón negro y va desde 1er dan hasta 6dan.
	\item los alumnos que anote un maestro, sus competidores, los equipos que se formen, los coachs que acompañan a los alumnos, etc etc, deben ser todos de la misma escuela.

\end{itemize}

\textbf{Escuela}(\uline{idEscuela}, nombre, pais, oro, plata, bronce, \dashuline{idPais})

Restricciones: La cantidad de bronce es menor o igual a la cantidad de matchs por el tercer puesto q se han disputados, y la cantidad de oro y plata es menor o igual a la cantidad de finales que se han jugado.

\textbf{Maestro}(\uline{nroPlacaInstructor}, nombre, apellido, graduacion, \dashuline{idEscuela}, )

\textbf{Pais}(\uline{idPais}, nombre)

\textbf{Alumno}(\uline{dniAlumno}, nombre , apellido, \dashuline{idEscuela} , \dashuline{dniCompetidor})

\textbf{InscripcionAlumno}(\uuline{nroPlacaInstructor} , \uuline{dniAlumno} , \dashuline{dniCompetidor})

\textbf{Coach}(\uuline{dniAlumno} , graduacion, nroCertITF, foto )

Restricciones: Debe haber un coach por cada 5 competidores.

\textbf{Competidor}(\uline{dniCompetidor}, nombre, apellido, fechaNacimiento, genero, graduacion, nroCertITF, peso, foto, \dashuline{idEscuela} , \dashuline{idEquipo} , \dashuline{idEquipo} , \dashuline{idCompetencia} )

Restricciones: Un competidor no puede ser titular y suplente a la vez, por lo tanto solo uno de los campos idEquipo puede tener contenido. Las competencias de las cuales participe su team, tienen que cumplir sus requisitos de edad y género.

\textbf{Equipo}(\uline{idEquipo} , nombreFantasia , \dashuline{idCompetencia} )

Restricciones: en cada Equipo hay 5 titulares y 3 suplentes, todos de la misma escuela.

\textbf{Competencia}(\uline{idCompetencia} )

\textbf{Single}(\uuline{idCompetencia}  )

\textbf{Modalidad}(\uline{idModalidad}, sexo, edad, graduacion, REVISAR )

\textbf{Forma}(\uuline{idModalidad}, graduacion )

\textbf{Combate}(\uuline{idModalidad}, graduacion , peso )

\textbf{Salto}(\uuline{idModalidad}, graduacion )

\textbf{Rotura de potencia}(\uuline{idModalidad}, graduacion )

\textbf{Team}(\uuline{idCompetencia} )

\textbf{ParticipaCompetidor}(\uuline{dniCompetidor}, \uuline{idModalidad}, \dashuline{idCompetencia})

Restricciones: el competidor tiene que cumplir los requisitos de edad, género , graduación y peso de la modalidad.

\textbf{Arbitro}(\uline{dniArbitro}, nombre , apellido , graduación , nroPlacaArbitro , \dashuline{idPais} )

\textbf{ArbitroPsteMesa}(\uuline{dniArbitro} , \dashuline(idRing) )

\textbf{ArbitroCentral}(\uuline{dniArbitro} , \dashuline(idRing))

\textbf{ArbitroJuez}(\uuline{dniArbitro} , \dashuline(idRing))

\textbf{ArbitroSuplente}(\uuline{dniArbitro} , \dashuline(idRing))

\textbf{ArbitrosEnCompetencias}(\uuline{dniArbitro} , \uuline{idCompetencia} )

\textbf{Ring}(\uuline{idRing})

Restricciones: un Ring siempre debe tener un presidente de mesa, un arbitro central , varios jueces y al menos 3 suplentes. la graduaciones de estos arbitros deben ser mayores a las competencias que se jueguen en ese ring (un 2 dan no puede arbitrar una competencia de 4to dan)

\textbf{Match}(\uuline{idMatch}, nroPelea , Ganador , Fecha , \dashuline{idRing} , \dashuline{idCompetencia})

\textbf{MatchIndividual}(\uuline{idMatch} )

\textbf{MatchGrupo}(\uuline{idMatch} )

\textbf{AsistenciaMatchIndividual}(\uuline{dniCompetidor}, \uuline{idMatch}, \dashuline{dniCoach(REVISAR)} )

Restricciones: coach y competidor no pueden ser la misma persona.

\textbf{AsistenciaMatchGrupo}(\uuline{idEquipo}, \uuline{idMatch}, \dashuline{dniCoach(REVISAR)} )

Restricciones: el coach no puede pertenecer al equipo.(esto es opinión mía, saquen conclusiones, cambienlo si no están de acuerdo, REVISAR)