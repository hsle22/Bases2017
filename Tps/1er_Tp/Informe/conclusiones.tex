\section{Conclusiones}

A lo largo del trabajo, nos encontramos con muchas cuestiones de decisión, en la clásica lucha claridad VS mantenimiento. Originalmente teníamos aproximadamente 24 entidades, que fueron recortadas a un total de 11, puesto que mantener la consistencia de tantas tablas resultaba mucho más costoso en nuestra opinión, que la perdida de claridad al tener menos entidades que almacenaran la misma información. Por otro lado, pudimos apreciar de primera mano como el pasaje del DER al MR y del MR a la base no es tan directo como la teoría podría sugerir, puesto que al realizar la base, nos hemos encontrado que cierta información resultaban muy complicadas de ser accedidas mediante querys por la estructura de la base, lo cual termino provocando en cambios en esta, que luego escalaron a cambios en el MR y obviamente en el DER. Finalizando, también experimentamos la complejidad de armar la estructura de una base en un conjunto de 4 personas, puesto que la diferencia de opiniones sobre un mismo enunciado, impacto gravemente en la definición de entidades, lo cual repercute eventualmente en la generación de la base.

En conclusión, consideramos que se alcanzaron los objetivos propuestos en este trabajo, puesto que pudimos toparnos y resolver de una manera que consideramos correcta, un ejemplo de la vida real sobre bases de datos.