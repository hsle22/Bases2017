\section{Introducción}

El objetivo de este segundo trabajo práctico es que, a partir de un dominio del problema ya tratado (Taekwon Do World Championship), se desarrolle una base de datos NoSQL para una extensión del mismo. A continuación se listan las funcionalidades de dicha base:

\begin{enumerate}

	\item La cantidad de enfrentamientos ganados por competidor para un campeonato dado.
	\item La cantidad de medallas por nombre de escuela en toda la historia.
	\item Para cada escuela, el campeonato donde ganó más medallas.
	\item Los arbitros que participaron en al menos cuatro campeonatos.
	\item Las escuelas que han presentado el mayor número de competidores en cada campeonato.
	\item Obtener los competidores que más medallas obtuvieron por modalidad.
	\item Experimentar con sharding. Crear para alguna tabla al menos tres shardings luego de haber insertado datos. Graficar la evolución de los shards a menida que se agreguen registros.
	\item Seleccionar una de las consultas e implementarla además utilizando Map-Reduce.
	
\end{enumerate}

Por otro lado, se pide que nuestra base tenga la capacidad de obtener las siguientes funcionalidades extra:

\begin{enumerate}

	\item Se desea guardar el historial de las competiciones mundiales.
	\item Para cada campeonato y cada categoría dentro de dicho campeonato se desean guardar todos los enfrentamientos con sus resultados, así como el medallero de cada uno.
	\item También se desea almacenar el histórico de árbitros y escuelas que han participado.

\end{enumerate}

 Además, la base utilizada cuenta con datos de prueba para poder probar el funcionamiento de las consultas incluidas en los requerimientos.