\section{Desarrollo}

%Dificultades surgidas en el tp, como se encararon, cambios que hubieron.
%Se nos pide:
Dado los requerimiento del enunciado tuvimos que agregar entidades, atributos a algunas de ellas y modificar algunas de sus relaciones, a continuación detallaremos los cambios realizados:
\bftext{Entidad Campeonato}
Para esto creamos la entidad Campeonato, que tendra los siguientes atributos:
\begin{itemize}
	\item id_campeonatos
	\item nombre
	\item fecha_inicio
	\item fecha_fin
\end{itemize}
y las siguientes relaciones:
\begin{itemize}
	\item escuelas
	\item arbitros
	\item competencias
\end{itemize}
Historial de Competencias: con la relación "competencias", cada campeonato guardará la información de las competencia que hubo en ella.

Historico de Arbitros: guardamos la asociacion de los arbitros que participaran en cada campeonato.

Historico de Escuela: guardamos la asociacion de las escuelas inscriptas en el campeonato.

\bftext{Entidad Competencia y Encuentro}
Para esto creamos la entidad Campeonato, que tendra los siguientes atributos:
Antes, en nuestro modelo guardabamos el puesto del competidor en la competencia que habia participado. Para cumplir con la nueva especificacion, hemos agregado la entidad "Encuentro", que guardará la siguiente información:
\begin{itemize}
	\item id_encuentro
	\item nro_encuentro: número de encuentro en la competencia con la modalidad correspondiente.
	\item competidorA: id del competidor
	\item competidorB: id del competidor
	\item ganador: id del competidor ganador 
\end{itemize}
de esta manera guardaremos la informacion de cada encuentro y puesto.
Tambien con esta informacion se puede obtener el medallero de cada competencia.


